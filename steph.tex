
1-Introduction:

For the possible twist on the paper: 
Not sure we should concentrate on the FOrbes et al 2011 paper, it is on dEs 
(real dEs, not transition) -in fact they found one that was more dIr/dE in their sample and 
ignored it in the rest of the paper. Our objects are real transitional objects closer to dIrs.
I think the better twist is along the lines of what was in the original proposal:
eg: several environmental effect can play a role in transforming a dIr into dE: galaxy harasment, tidal stirring, ram-pressure stripping. The abundance gradients & vel disp will bring clues to what physical event might be responsible for transformation.

2- Observations

We obtained long-slit spectroscopy of our 4 nearby transition-type dwarf galaxies using GMOS 
on Gemini-South with queue-observing (GS-2008A-Q-44, Band 2). The data were obtained over 
a period of less than a month in March and April 2008. In each case the longslit was positionned 
along the major axis of the galaxies (see Figure x).
The B1200 grating was used, with a 1" slit width, and was centered at 4700 \AA with 
a wavelength coverage from 3950 \AA to 5410 \AA at 0.47 \AA per pixel, giving a resolution 
of $\lambda/\Delta \lambda = 1872$.
The GMOS-South detector array consists of 3 EEV CCDs of 2048(spectral)$\times$4068(spatial) pixels 
detectors, with pixel size of 0.073", which were binned by 2(spectrally)$\times$4(spatially).
Each pair of integrations were observed with a dither of 50\AA to be able to recover a full coverage 
of the wavelength region over the CCD gaps between the three detectors which are about 39 unbinned pixels
 (although the regions in the gap received all in all only half the total integration times and have 
therefore lower S/N).
The total integration times were 2$\times$1800 sec for NGC5237, 2$\times$3600 sec for 
ESO272-G025, and 6$\times$3600 sec each for ESO384-G016 and ESO269-G058. We also acquired internal 
Quartz-Halogen lamps flatfields every hour, one CuAr arc for wavelength calibration the same day of the observations, 
one spectroscopic standard per galaxy from the Strobel catalogue with known effective 
temperature, metallicity and radial velocity, to measure the instrumental response and test 
the asymmetry and width of the line-spread function, as well as one twilight flatfield with high S/N.

3- Data Reductions

The data were reduced using mostly the Gemini iraf package. Bias subtraction was done using 
mean bias frames created by combining all available bias frames of each of the GMOS-S observing runs. 
A bad pixel mask was constructed using the Quartz-Halogen flats. The data were flatfielded using 
the QH flats taken either just before or after the science exposures. Cosmic rays were removed using the 
Laplacian Cosmic Ray identification algorithm of van Dokkum (2001). The wavelength calibration was 
determined using the CuAr arcs spectra. The spectra were then sky-subtracted, and multiple exposures were 
combined to produce one final 2D spectrum per galaxy.
